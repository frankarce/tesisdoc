%\input{preambulo.tex}

%\input{portada.tex}
%\input{frontmatter.tex}

%Tamaño carta, 12 puntos

%Estilo tesis de un solo lado, cada capitulo empieza en la
\documentclass[12pt,letterpaper,oneside] {memoir}  
% Use utf-8 encoding para que acepte acentos etc.
\usepackage[utf8x]{inputenc}
%Para Graficos
\usepackage{graphicx}
%Para la inscripción con guión en español
%\usepackage[spanish]{babel}
\usepackage[spanish]{babel}
\usepackage{float}
\usepackage{xcolor,calc}
%Para algoritmos
\usepackage{amsthm}
\usepackage{amssymb,amsmath}
\usepackage{multicol}
%%%%%
%%%%%Editen el archivo texmf/tex/generic/config/laguange.dat
%%%%%Borren el % que esta la principio de la linea spanish. Si mal no recuerdo viene preparado con silabeo en alemán, si no lo quieren pongan un % adelante. También viene predefinido inglés (o alguna de sus variantes, por ejemplo american) hasta donde sé este tiene que estar definido así que no lo saquen.
%%%%%
%%%%%
\usepackage[boxed]{algorithm} 
%\usepackage{algorithm}
%\usepackage{algorithmic}
%\usepackage{algorithmicx}

\usepackage{algpseudocode}
% A doble espacio de renglon
\linespread{1.6}\selectfont



%Doble espacio
%% Times New Roman
%\renewcommand{\rmdefault}{ptm}


%%Palatino
%\usepackage[T1]{fontenc}
%\usepackage[sc]{mathpazo}
%\usepackage{lmodern}

%%%Concrete
%\usepackage{ccfonts,eulervm}
%\usepackage[T1]{fontenc}


%%CM-Bright
%\usepackage{cmbright}

%% Vera
%\usepackage[T1]{fontenc}
%\usepackage{bera}

%%%Garamond
%\usepackage[T1]{fontenc}
%\usepackage[urw-garamond]{mathdesign}
%\usepackage[garamond]{mathdesign}


%\usepackage{kmath,kerkis}
%\usepackage{tgbonum}

%\usepackage{fourier}

\settocdepth{subsection}

%\usepackage[T1]{fontenc}
%\usepackage[charter]{mathdesign}
\setsecnumdepth{subsection}
\maxsecnumdepth{subsection}
\settocdepth{subsection}
\maxtocdepth{subsection}


%\usepackage[plain]{flexbib}
%\usepackage[numbers,square]{natbib}
%\usepackage[numbers]{natbib}
%\setcitestyle{numbers}
%\citestyle{plain}
%\bibliographystyle{apa-good} %cambiar el estilo a num sort. estilo para numeros y apellidos.
%\bibliographystyle{apa-good} 
%\bibpunct{[}{]}{;}{a}{,}{,} %Cambio parentesis x corchete
%Estilo de la bibliogr2afía
%\bibliographystyle{apalike}
%\bibliographystyle{newapa}

\usepackage[numbers]{natbib}
%\bibliographystyle{apa-good}
%\bibpunct{(}{)}{;}{a}{,}{,}


\settrimmedsize{11in}{215mm}{*} %210 mueve el texto.
\settrims{0in}{0in}
\settypeblocksize{8.65in}{37pc}{*} %8.5 x 38a la mitad ancho del texto en la hoja.
\setlrmargins{1.2in}{*}{1in} % izquierdo y derecho 1.3in
\setulmargins{3cm}{*}{*} %arriba
\setheadfoot{12pt}{21pt} %26 o 24 pt
%\setheaderspaces{*}{13pt}{*}
\setheadfoot{\onelineskip}{2\onelineskip} 
\setheaderspaces{*}{2\onelineskip}{*} 

\checkandfixthelayout


%% Nicely format and linebreak URLs in the bibliography (and elsewhere).
\usepackage{url}
\newcommand{\HRule}{\rule{\linewidth}{0.5mm}}
\usepackage{listings}
\usepackage{pifont}
\usepackage{bbding}

%\usepackage[papersize={841mm,1189mm},lmargin=2cm,rmargin=2cm,top=2cm,bottom=2cm]{geometry}

%%% BEGIN DOCUMENT
\begin{document}
	\thispagestyle{empty} 
\enlargethispage*{1000pt}
	\begin{center} 
	{ \textsc{\Large SEP } \hfill  \textsc{\Large DGEST}}\\[0.5cm]
	%\textsc{\Large SEP DGEST}\\[1.0cm]
	\textsc{\Large Instituto Tecnológico de Tijuana}\\[0.5cm]
	\textsc{\Large División de Estudios de Postgrado e Investigación }\\[1.0cm]
	% Upper part of the page
	\includegraphics[width=0.3\textwidth]{./logo}\\[0.5cm] %0.15
	%\textsc{\large Maestría en Ciencias en Ciencias de la Computación}\\[1.0cm]
	% Title
	%\HRule \\[.2cm]
	{ \LARGE \bfseries SECUENCIADO ADAPTATIVO DE OBJETOS DE APRENDIZAJE EN AMBIENTES INTELIGENTES}\\[1.5cm]
	%\HRule \\[.2cm]
	\begin{minipage}{1.0\textwidth}
	\begin{flushright} 
	\textsc{ Trabajo de tesis}\\[0.5cm]
	\emph{Presentado por:} \\
	% Author and supervisor
	%\vspace{5 mm}
	%\emph{}
	\textsc{ Francisco Javier Arce Cárdenas}	\\
	\vspace{5 mm}
	\emph{Para obtener el grado de:} \\
	\textsc{Doctor en Ciencias  de la Computación} \\
	\vspace{5 mm}
	\emph{Director:} \\
	\textsc{Dr. José Mario García Valdez} \\
	\vspace{5 mm}
	\emph{Tijuana, B.C. XXXXXX del 20XX.}
	\end{flushright}
	\end{minipage}
	\vspace{13 mm}
	\vfill
	% Bottom of the page

	\end{center}
\clearpage
\thispagestyle{empty}
%\begin{flushright}
%\textit{Para mi familia y amigos.}
%\end{flushright}

%%%%Inicio número de páginas romanos
%Traduccion de los nombres 
\renewcommand{\chaptername} {Capítulo}
\renewcommand{\abstractname} {Resumen}
%\renewcommand{\bibname} {Bibliografía}
\renewcommand{\contentsname} {Contenido}
\renewcommand{\figurename} {Figura}
\renewcommand{\listfigurename} {Lista de Figuras}
\renewcommand{\listtablename} {Lista de Tablas}
\renewcommand{\figurename} {Figura}
\renewcommand{\tablename} {Tabla}
\renewcommand{\appendixname} {Apéndice}
%\renewcommand{\listequationname} {Lista de Ecuaciones}
%\renewcommand{\equationname} {Ecuación}


\frontmatter
\begin{abstract}
	
\end{abstract}

\clearpage

\renewcommand{\abstractname} {Abstract}
\begin{abstract}
Learning environments are places or spaces where the learning process is given; they can be classrooms, museums and etcetera.  Instead of intelligent learning environments besides being places where the process of learning takes place using devices that help improve this process. The update and emergence of new technologies changed the traditional methods for learning in intelligent learning environments, as interactive tables, more powerful smartphones, tablet pcs, cameras and cameras that perceive depth (kinect). In this work we propose a new approach and the personalization of learning objects that we will call environmental learning object which can be interchanged between (adapted to) different learning environments. These learning objects will embody packages of didactic resources together with rules for device assignment and composition and deploy them on an intelligent learning environment in which we will use a single extension of the simple sequencing standard.
\end{abstract}

\clearpage

\section*{Agradecimientos\markboth{}{}}
\paragraph{}
...
\paragraph{}
...
\paragraph{}
...
\paragraph{}
....
\paragraph{}
...
\paragraph{}
....
\paragraph{}
...
\clearpage

\tableofcontents* 
\clearpage
\listoffigures
\clearpage
\listoftables
%\listofequation
%\listofalgorithms
%inicia el documento, con número de páginas arabigos
\mainmatter

\chapter{Introducción}

A learning environment is defined as a \textit{place} or \textit{space} where the process of knowledge acquisition occurs. Learning environments are those areas where conditions are created for the individual to appropriate new knowledge, new experiences, new elements that generate processes of analysis, reflection and appropriation.  
\paragraph{}
Some of these systems use learning resources called learning objects. A learning object defined by \cite{Willey 2000} as digital or non- digital entities, which can be used to promote learning, education or entertainment. Standard initiatives have been proposed to promote the exchange of learning objects between learning management systems (LMS?s) these are used in repositories, and several systems \cite{LMS 2}.   
\paragraph{}
For the exchange of learning objects between systems standardization initiatives have been developed and there are some implementations and repositories that manage the content using these standards. Learning objects usually have the following characteristics are self-contained, each learning object can be used independently, are reusable, a learning object can be used in different contexts, can be added, learning objects can be grouped into collections and are labeled with metadata.   
\paragraph{}
Each learning object has associated certain information that describes it. This facilitates reuse by automatic means. Standard initiatives have been proposed to promote the exchange of learning objects between learning management systems (LMS?s) these are used in repositories, and several systems [2]. Different techniques (discussed later) have been proposed to personalize the sequencing and selection of learning objects to accommodate the learner?s individual requirements. Adaptive hypermedia (AH?s) techniques have been used successfully in Web based educational interactive systems, but these systems are mainly limited to browser-based interactions. Traditional interaction devices like Desktop PCs and laptop computers, are now joined by other devices as described by Poslad in his UbiComp model [3]:
\begin{itemize}
\item Smart Devices. Multifunctional, mobile, personalized, private, to ease access to and embody services rather than just to virtualize them.
\item Smarter Environments. To sense and react to events such as people, with mobile devices, entering and leaving controlled spaces.
\item Smarter Interactions. These use other service access devices with simpler functions and allow them to interoperate between devices.
\end{itemize}
\paragraph{}
These devices can be used to compose Intelligent Environments (IEs) defined as physical environments in which information and communication technologies and sensor systems disappear as they become embedded into physical objects, infrastructures, and the surroundings in which we live, travel, and work. Context awareness plays a key role, in these systems, as the environment intelligently has to perceive users, devices, the physical environment and their interactions. There are currently several proposals using this kind of devices for educational purposes, integrated with intelligent environments, personalization and context awareness [4][5].
\paragraph{} 
There are other proposals that do not consider the use of learning objects nor reuse of didactic content in intelligent environments, in this work we propose an extension to learning objects techniques that could be interchanged between (adapted to) different learning environments. These learning objects will embody packages of didactic resources together with rules for device assignment and composition. We intend to use adaptive hypermedia systems techniques and learning objects in the development of intelligent learning environments. 
\paragraph{}
\section{Thesis Structure}
This thesis is organized as follows:
\begin{itemize}
\item \textit{Chapter 2} The theory and background are presented.
\item \textit{Chapter 3} Proposed framework are presented. 
\item \textit{Chapter 4} Describes the application problems reviewed in this investigation work.
\item \textit{Chapter 5} The experimental results are shown.
\item \textit{Chapter 6} Conclusions and current/future work are presented.
\item \textit{Chapter 7} Presents Annex A
\end{itemize}
\chapter{Theory and Background}
This chapter overviews the background and main definitions and basic concepts, useful to the development of this investigation work.
\paragraph{} 
\section{Learning environment}
Learning environment refers to the diverse physical locations, contexts, and cultures in which students learn. Since students may learn in a wide variety of settings, such as outside-of-school locations and outdoor environments, the term is often used as a more accurate or preferred alternative to classroom, which has more limited and traditional connotations?a room with rows of desks and a chalkboard, for example.
\paragraph{}
The term also encompasses the culture of a school or class?its presiding ethos and characteristics, including how individuals interact with and treat one another?as well as the ways in which teachers may organize an educational setting to facilitate learning?e.g., by conducting classes in relevant natural ecosystems, grouping desks in specific ways, decorating the walls with learning materials, or utilizing audio, visual, and digital technologies. And because the qualities and characteristics of a learning environment are determined by a wide variety of factors, school policies, governance structures, and other features may also be considered elements of a  \textit{learning environment}.
\paragraph{} 
Educators may also argue that learning environments have both a direct and indirect influence on student learning, including their engagement in what is being taught, their motivation to learn, and their sense of well-being, belonging, and personal safety. For example, learning environments filled with sunlight and stimulating educational materials would likely be considered more conducive to learning than drab spaces without windows or decoration, as would schools with fewer incidences of misbehavior, disorder, bullying, and illegal activity. How adults interact with students and how students interact with one another may also be considered aspects of a learning environment, and phrases such as \textit{positive learning environment} or \textit{negative learning environment} are commonly used in reference to the social and emotional dimensions of a school or class.
\paragraph{}
\section{Interactive learning environment}
 Interactive learning is a pedagogical approach that incorporates  social networking and urban computing into course design and delivery. Interactive Learning has evolved out of the hyper-growth in the use of digital technology and virtual communication, particularly by students. 
\paragraph{}
The use of interactive technology in learning for these students is as natural as using a pencil and paper were to past generations. The Net Generation or Generation Y is the first generation to grow up in constant contact with digital media. Also known as digital natives, their techno-social, community bonds to their naturalized use of technology in every aspect of learning, to their ability to learn in new ways outside the classroom, this generation of students is pushing the boundaries of education. The use of digital media in education has led to an increase in the use of and reliance on interactive learning, which in turn has led to a revolution in the fundamental process of education.\paragraph{}
Algunos ejemplos de recolección de datos de \textbf{forma explícitas} son:
\paragraph{}
Increasingly, students and teachers rely on each other to access sources of knowledge and share their information, expanding the general scope of the educational process to include not just instruction, but the expansion of knowledge. The role change from keeper of knowledge to facilitator of learning presents a challenge and an opportunity for educators to dramatically change the way their students learn. The boundaries between teacher and student have less meaning with interactive learning.
\section{Intelligent learning environment}
An intelligent learning environment is a new kind of intelligent educational system, which combines the features of traditional Intelligent Tutoring Systems (ITS) and learning environments. An intelligent learning environment (ILE) includes special component to support student-driven learning, the environment module. The term environment is used to refer to that part of the system specifying or supporting the activities that the student does and the methods available to the student to do those activities [8]. Some recent ITS and ILE include also a special component called manual which provides an access to structured instructional material. The student can work with the manual via help requests or via special browsing tools exploring the instructional material on her own. An integrated ILE, which includes the environment and the manual components in addition to regular tutoring component, can support learning both procedural and declarative knowledge and provide both system-controlled and student-driven styles of learning.
\paragraph{}
\section{Fuzzy Logic}
Zadeh introduced the term fuzzy logic in his work ?fuzzy sets?, where he described the mathematics of the fuzzy set theory in 1965.
Fuzzy logic gives the opportunity to model conditions that are defined with imprecision.
\paragraph{}
The tolerance of the fuzzy in the process of human rezoning suggests that most of the logic behind the human rezoning is not the traditional bi-valued logic, or even the multi-valued, but the logic with fuzzy values, with fuzzy connections and fuzzy rules or inferences.
\paragraph{}
\subsection{Fuzzy sets}
Fuzzy sets are an extension of the classic set theory and, as it name implies it, it is a set with boundaries not well defined, this means that the transition of belonging or not belonging to certain set is gradual, and this smooth transition is characterized by grades of membership that gives the fuzzy sets flexibility in modeling linguistic expressions commonly used, such as ?the weather is cold? or ?Gustavo is tall?.
\paragraph{}
\subsection{Fuzzy logic controller}
Fuzzy control is a control method based on fuzzy logic. Just as fuzzy logic can be described simply as ?computing with words rather than numbers? fuzzy control can be described simply as ?control with sentences rather than equations?.
\paragraph{}
The collection of rules is called a rule base. The rules are in the familiar if-then format, and formally the ?if? side is called the antecedent and the ?then? side is called the consequent.
\paragraph{}
Fuzzy controllers are being used in various control schemes; the most used is the direct control, where the fuzzy controller is in the forward path in a feedback control system. The process output is compares with a reference, and if there is a deviation, the controller takes action according to the control strategy.
\paragraph{}
In a feed forward control a measurable disturbance is being compensated, it requires a good model, but if a mathematical model is difficult or expensive to obtain, a fuzzy model may be useful. Fuzzy rules are also used to correct tuning parameters. If a nonlinear plant changes operating point it may be possible to change the parameters of the controller according to each operating point. His is called gain scheduling since it was originally used to change process gains. 
\paragraph{}
A gain scheduling controller contains a linear controller whose parameters are changed as a function of the operating point in a preprogrammed way. It requires thorough knowledge of the plant, but it is often a good way to compensate for nonlinearities and parameter variations. Sensor measurements are used as scheduling variables that govern the change of the controller parameters, often by means of a table look-up.
\paragraph{}
\section{Learning Object}
A learning object is "a collection of content items, practice items, and assessment items that are combined based on a single learning objective".[1] The term is credited to Wayne Hodgins when he created a working group in 1994 bearing the name[2] though the concept was first described by Gerard in 1967.[3] Learning objects go by many names, including content objects, chunks, educational objects, information objects, intelligent objects, knowledge bits, knowledge objects, learning components, media objects, reusable curriculum components, nuggets, reusable information objects, reusable learning objects, testable reusable units of cognition, training components, and units of learning. Learning objects offer a new conceptualization of the learning process: rather than the traditional "several hour chunk", they provide smaller, self-contained, re-usable units of learning.
\paragraph{} 
They will typically have a number of different components, which range from descriptive data to information about rights and educational level. At their core, however, will be instructional content, practice, and assessment. A key issue is the use of metadata.
Learning object design raises issues of portability, and of the object's relation to a broader learning management system.
\paragraph{}
\begin{figure}[H] 
 \centering 
\framebox{
\includegraphics[width=0.6\textwidth]{./images/procesoRecomendacion2.pdf} 
} \caption{Esquema de proceso de generación de una recomendación.} 
 \label{fig:procesoRecomendacion} 
\end{figure}
\chapter{Related Work}
\paragraph{}
\item[Zagat.] Es una empresa americana fundada en 1979 por Tim y Nina Zagat que se dedica a la edición de todo tipo de guías de restaurantes, hoteles, clubes o tiendas de distintas ciudades de los Estados Unidos y Canadá \citep{Resnick1997}. En \textit{www.zagat.com} los usuarios registrados pueden votar distintos aspectos del local preferido y, además, introducir pequeños comentarios con su experiencia. En base a estas votaciones los responsables de la empresa asignan su puntuación en sus guías anuales y hacen recomendaciones individuales a sus usuarios a través de su web. Estas recomendaciones son hechas por un sistema de recomendación  colaborativo. Zagat es la página Web comercial que utiliza un sistema de recomendación colaborativo, donde las valoraciones de los restaurantes son actualizadas constantemente por los usuarios. Incluye muchas opciones de búsquedas avanzadas donde el usuario puede especificar sus criterios de búsqueda en categorías tales como comida, decoración, costos, etcétera, o restringir la búsqueda a cierta área o un tipo de cocina concreta.
\end{description}

Enseguida se muestra un cuadro comparativo \citep{Resnick1997} de los sistemas de recomendación mencionados.

\begin{table}[H]
\begin{small}
\caption{Cuadro comparativo de sistemas de recomendación.} \begin{center}
\linespread{0.9}\selectfont
\begin{tabular}{>{\footnotesize}p{0.8in}>{\footnotesize}p{0.8in}>{\footnotesize}p{0.8in}>{\footnotesize}p{0.8in}>{\footnotesize}p{0.8in}>{\footnotesize}p{0.8in}}
\hline                  
         & El contenido de la recomendación & ¿La entrada es explícita? &  ¿Es anónimo? & Modo de agregación  & Uso de las recomendaciones \\
\hline             
Fab  & Numérica: 1-7. & Explícita. & Pseudónimo. & Ponderación personalizada; combinada con análisis de contenido.  & Selección/filtrado. \\

ReferralWeb & Mención de una persona o un documento. & Extraídas de fuentes públicas de datos. & Atribuído.  & Montar cadena de referencias a la persona deseada.  & Pantalla. \\

PHOAKS & Mención de una URL. & Extraídos de publicaciones Usenet.  & Atribuído.  & Una persona, un voto (por URL). & Ordenados en pantalla. \\

Siteseer  & Mención de una URL. & Extraídos de las carpetas de marcadores existentes. & Anónimo. & Frecuencia de mención en la superposición de las carpetas. & Pantalla.\\

Amazon & Numérica:1-5. & Explícita. & Pseudónimo.  & Por cliente, valora todos los productos comprados. &  Pantalla.  \\

Zagat  & Numérica: 0-3. & Explícita. & Pseudónimo. & Por usuario, valora todos los aspectos de un establecimiento. & Pantalla, sugerencias de los demás usuarios.  \\

\hline
\end{tabular}
\label{tab:cuadro}
\end{center}
\end{small}
\end{table}

\chapter[Recomet: Recomendaciones de Tijuana]{Recomet}

En este capítulo se presenta un prototipo de sistema de recomendación de restaurantes para la ciudad de \textbf{Tijuana}, está basado en técnicas de lógica difusa y trabaja con un mecanismo de hibridación por pesos para generar recomendaciones.
 \paragraph*{}   
El sistema  recomendación híbrido está  compuesto de dos sistemas de  recomendación, uno colaborativo y uno basado en contenido.  En este sistema de recomendación, se ha empleado  un  mecanismo de \textbf{hibridación por pesos} \citep{Burke2002}. En este tipo de sistemas híbridos las técnicas de recomendación trabajan simultáneamente y para este caso, es un sistema difuso el que asigna los pesos a cada recomendación generada en el sistema para mostrar al usuario un promedio ponderado.
\paragraph{}
El  objetivo fue crear un sistema de recomendación que se pudiera aplicar en situaciones en donde a los usuarios les gustaría recibir recomendaciones sobre algún restaurante para comer. Implementar sistemas de recomendación de restaurantes, lugares turísticos o sitios de ocio resulta un poco complicado ya que presentan problemas que son considerables a la hora de generar recomendaciones \citep{PerezCordon2008}, enseguida se mencionan:

\begin{enumerate}
\item Muchos de los usuarios que interaccionan con el sistema, son \textbf{usuarios casuales} que nunca han usado el sistema de recomendación o que lo han usado de forma esporádica y no piensan utilizarlo de forma habitual. En el capítulo 2  se explicó que las técnicas clásicas de recomendación sufren del problema del nuevo usuario y por lo tanto no son capaces de generar recomendaciones cuando se encontraban en estas situaciones.
\item Un número importante de usuarios tendrá un \textbf{conocimiento basado en expectativas} sobre el servicio o producto que quieren recibir y es muy probable que no sepan expresar de forma clara y precisa las características del tipo de restaurante que desean visitar.
\item Es habitual que en este tipo de situaciones, existan usuarios que quieran recibir \textbf{recomendaciones puntuales} que no tengan nada que ver con lo que han hecho en el pasado. Por lo tanto, en estos casos, la información histórica no será relevante y no debería ser utilizada en la generación de recomendaciones. Por ejemplo, si un cliente desea celebrar un cumpleaños en un restaurante, es muy probable que sea un hecho puntual y que no quiera que se use la información de los restaurantes que le han gustado en el pasado para generar estas recomendaciones.
\end{enumerate}
Sin embargo, pese a este tipo de situaciones las recomendaciones generadas han demostrado ser aceptables.
\paragraph*{}
El prototipo en general fue desarrollado en Lenguaje \textbf{Python} \citep{Drake2000}, utilizando el framework \textbf{Django v1.2.} \citep{Django2008} para el desarrollo Web,  este framework permitió la implementación  sencilla de los algoritmos de recomendación integrados en el sistema (ver figura \ref{fig:arquitectura}), Django proporciona diversas funcionalidades que minimizan el esfuerzo de desarrollo. \textbf{Django} se complementa con el lenguaje de etiquetas \textbf{HTML} para el diseño de las páginas Web. 
La base de datos fue diseñada en \textbf{Visual Paradigm for UML v4.2} y creada en \textbf{PostgreSQL v9.0.} y un servidor Web para instalar la aplicación (Apache). 
En la  figura \ref{fig:p2} se muestra el prototipo final elaborado para pruebas y algoritmos, el principal objetivo de esta investigación.

\section{Aplicación}

En esta sección se describe cada función especificada en el prototipo, que inicia desde la parte administrativa hasta llegar a la recomendación para el usuario.

\subsubsection{Administración}
La parte inicial del sistema fue elaborar el sistema de administración basado en el modelo que ofrece el framework \textbf{Django},  fue sencillo de implementar por su funcionalidad, así, la administración del sistema se realiza desde la interfaz integrada en el framework, y se muestra en la figura  \ref{fig:p1}.

\begin{figure}[H]
\centering 
\framebox{
\includegraphics[width=0.7\textwidth]{./images/p1.pdf} 
} \caption{Administración del sitio.} 
\label{fig:p1} 
\end{figure}

Desde aquí la tarea de administrar el sitio se hace sencilla, en la figura \ref{fig:p1} se encuentran los \textbf{usuarios} (Users), los \textbf{grupos de usuarios} (Groups), los \textbf{contenedores}, los \textbf{restaurantes} (Items) de la base de datos, las \textbf{valoraciones} (Ratings) y todos los demás elementos que el usuario administrador debe controlar.  Desde aquí se manipula toda la información contenida en la base de datos, todos los detalles como fechas, horas, acciones de los usuarios, etcétera, ya que el administrador cuenta con todos los permisos para realizar cualquier transacción.

\subsubsection{Funcionalidad}

Para obtener recomendaciones, en primera instancia es importante recalcar que solamente los usuarios registrados obtendrán recomendaciones de este sistema como se observa en la figura \ref{fig:p2}, existe el \textbf{usuario invitado} (Guess) que solo podrá visualizar la información de los demás usuarios, las reseñas, perfiles de restaurantes y demás pero en ningún caso podrá agregar algún tipo de información.\\

\begin{figure}[H]
\centering 
\framebox{
\includegraphics[width=0.8\textwidth]{./images/p2.pdf} 
} \caption{Pantalla principal del prototipo.} 
\label{fig:p2} 
\end{figure}

Para obtener los datos sobre un restaurante el usuario invitado debe pulsar sobre la opción \textbf{Restaurants} que aparece en la parte superior derecha de la pantalla principal. Una vez pulsada aparecerá la pantalla que se muestra en la figura \ref{fig:p3} donde se muestra información detallada en el perfil de cada restaurante.\\

\begin{figure}[H]
\centering 
\framebox{
\includegraphics[width=0.8\textwidth]{./images/p3.pdf} 
} \caption{Listado de restaurantes para el usuario invitado.} 
\label{fig:p3} 
\end{figure}

Por otra parte, cuando el usuario esta registrado y es autentificado en el sistema aparecerá una sección de recomendaciones \textbf{(Recommendations for you)} que surgen en base a los algoritmos implementados en el sistema, esta es la respuesta a las preferencias que el usuario activo dio al sistema al realizar las valoraciones de los restaurantes, \textbf{las sugerencias} son mostradas como resultado del algoritmo colaborativo y el basado en contenido, así como la \textbf{recomendación del experto} basado en los parámetros explicados en el capítulo 3.
\paragraph{}
Se agregó la liga para acceder a la lista de restaurantes de la base de datos del sistema, en la figura \ref{fig:p4} se muestran los perfiles de cada restaurante, a su vez cada perfil tiene las opciones para  ver:
\begin{itemize}
\item Las valoraciones de otros usuarios que han votado ese restaurante.
\item Las reseñas que han agregado de su experiencia en ese restaurante.
\item Una ventana donde aparecerá la opción para votar ese restaurante.
\item Un botón para agregar el restaurante a una \textbf{lista de interés} (wishlist) si aún no lo ha visitado.
\end{itemize}

\begin{figure}[H]
\centering 
\framebox{
\includegraphics[width=0.8\textwidth]{./images/p4.pdf} 
} \caption{Información de perfiles de restaurantes para el usuario registrado.} 
\label{fig:p4} 
\end{figure}

Ciertamente, en la liga para \textbf{My Wishlist} mostrada en la parte superior izquierda de la página, se encuentran todos los restaurantes que el usuario activo ha seleccionado como restaurantes de su interés. La figura \ref{fig:p5} muestra la pantalla que aparece una vez que se ha pulsado en la liga.\\

\begin{figure}[H]
\centering 
\framebox{
\includegraphics[width=0.8\textwidth]{./images/p5.pdf} 
} \caption{Lista de interés del usuario activo.} 
\label{fig:p5} 
\end{figure}

Desde la página principal también se observa la sección del perfil del usuario activo (\textbf{xochilt’s profile} para este caso), en esta sección se encuentra la descripción del perfil del usuario, se define en base a los datos proporcionados por él mismo, cabe mencionar que desde ahí el usuario tiene el control de los restaurantes que le interesan visitar, los que le gustan y los que le desagradan, también puede visualizar desde las ligas las valoraciones que ha asignado a los restaurantes que evaluó \textbf{(ratings)} y visualizar las reseñas que ha descrito sobre los restaurantes que ha visitado \textbf{(reviews)} ofreciendo también la posibilidad de agregar una nueva reseña \textbf{(add review)}, como se observa en la figura \ref{fig:p6}.

\begin{figure}[H]
\centering 
\framebox{
\includegraphics[width=0.8\textwidth]{./images/p6.pdf} 
} \caption{Perfil del usuario activo.} 
\label{fig:p6} 
\end{figure}

Otra opción que ofrece a los usuarios el sistema, es que pueden agregar restaurantes a la base de datos desde la liga \textbf{Add restaurant}  que se muestra en el menú principal (ver la figura \ref{fig:p2}).
\paragraph{}
Esta opción también es manejada desde el administrador, se pueden agregar restaurantes y sólo el administrador tiene permisos para aceptar los restaurantes agregados y eliminarlos si los datos están incorrectos, \textbf{Django administration} ofrece esta funcionalidad y se muestra en la figura \ref{fig:p7}.

\begin{figure}[H]
\centering 
\framebox{
\includegraphics[width=0.8\textwidth]{./images/p7.pdf} 
} \caption{Manipulación de información de restaurantes desde el administrador.} 
\label{fig:p7} 
\end{figure}


\subsubsection{Módulo de filtrado colaborativo}

El módulo de filtrado colaborativo aplicado en el sistema \citep{PerezCordon2008} se resume en tres pasos:
\begin{enumerate}
\item El sistema \textbf{guarda un perfil} de cada usuario con sus evaluaciones sobre los restaurantes.
\item Se mide el \textbf{grado de similitud} entre los distintos usuarios del sistema en base a sus perfiles y se crean grupos de usuarios con características afines.
\item El sistema usará toda la información obtenida en las fases anteriores para \textbf{realizar las recomendaciones}. A cada usuario, le recomendará restaurantes que no haya evaluado y que hayan sido evaluados de manera positiva por el resto de miembros de su grupo.
\end{enumerate}

Gráficamente el sistema no muestra una sección para ver las recomendaciones basadas en filtrado colaborativo. Este proceso es interno y transparente al usuario así como los resultados obtenidos de esta técnica. Sin embargo, cabe mencionar que la operación de recomendación completa y de acuerdo a la arquitectura descrita en el capítulo 3,  se encuentra con la combinación de los tres métodos utilizados para generar recomendaciones al usuario activo. 
\paragraph{}
Esta sección se encuentra desde el menú principal (ver figura \ref{fig:p2}) en la liga \textbf{Recommendations for you}, donde se muestran las sugerencias.

\subsubsection{Módulo basado en contenido}

De igual manera que el módulo de filtrado colaborativo, el módulo basado en contenido está integrado en esta recomendación final. El proceso de este algoritmo se ha explicado en el capítulo 3, por lo tanto se reserva la explicación completa del proceso, sólo se explicará los detalles relevantes sobre este algoritmo en la aplicación.
\paragraph{}
Es importante mencionar que en este caso particular, el algoritmo ha trabajado con vectores binarios para la representación de las características de los restaurantes, el proceso es simple a la hora de comparar con la \textbf{similaridad de cosenos} para obtener  los restaurantes \textbf{más parecidos}, esta técnica también implementada en \citep{Burke2002}, mostró buenos resultados.  
\paragraph{}
Existen otras maneras como en \citep{Cristian2010} donde se muestra que la similaridad es basada en la repetición de palabras en un documento, utilizando principalmente la técnica de \textbf{Frecuencia de Términos y Frecuencia del Documento Inversa} (TF/IDF por sus siglas en inglés) \citep{Adomavicius2005} para obtener valores de aparición de cada palabra.  
Se puede comparar un documento Web con un perfil de restaurante tratándolos como documentos ambos,  pero en este sistema no se aplicó esta técnica porque los perfiles de los restaurantes son documentos pequeños y basarlos en las palabras que tienen mayor aparición sería inadecuado. En la figura \ref{fig:p8} se muestra el documento del perfil de restaurante que surge como una recomendación para el usuario activo, basada en los dos algoritmos.\\

\begin{figure}[H]
\centering 
\framebox{
\includegraphics[width=0.8\textwidth]{./images/p8.pdf} 
} \caption{Recomendación final para el usuario activo.} 
\label{fig:p8} 
\end{figure}


\chapter[Métodos de evaluación]{Métodos de evaluación}

\section{Métricas}

Evaluar un sistema de recomendación debería ir mas allá de la \textbf{precisión} \citep{Adomavicius2005, Herlocker1991}, es lo más adecuado si se requiere que el sistema logre recomendaciones que realmente sean útiles al usuario. Por ejemplo, en informática una recomendación podría obtener un alto valor, pero en realidad ser una recomendación solo basada en la popularidad de los ítems, llegando a ser una sugerencia inútil.
\paragraph{}
Una de las métricas más convenientes para evaluar las recomendaciones es la idoneidad \citep{Herlocker2004} que incluye:
\begin{itemize}
\item \textbf{La cobertura} que mide el porcentaje de un conjunto de datos donde un sistema de recomendación es capaz de generar prediciones, 
\item \textbf{Indicadores de confianza} que pueden ayudar a los usuarios a tomar mejores decisiones.
\item \textbf{La tasa de aprendizaje} que mide la rapidez con que un algoritmo puede generar buenas recomendaciones.
\item \textbf{La novedad} que mide si una recomendación es una nueva posibilidad para un usuario.
 \end{itemize}
 
\paragraph{}
Por otra parte hay otros métodos estadísticos que sólo miden \textbf{la precisión} de los algoritmos en la recomendación, es evidente que una evaluación eficaz del rendimiento de los algoritmos de recomendación no es tarea sencilla. Primero porque diferentes algoritmos pueden ser mejores o peores dependiendo del conjunto de datos elegido. La mayoría de estos algoritmos están diseñados para el conjunto de datos de películas de \textbf{GroupLens} donde existe un número de elementos mucho menor al número de usuarios y votos. 
\paragraph{}
En una situación inversa el comportamiento puede diferir totalmente. El principio de este problema es que realmente no existe un método estandarizado que permita el logro de evaluaciones realmente significativas para estos algoritmos.
\paragraph{}
También si se consideran los objetivos del sistema de recomendación éstos pueden diversos. Un sistema puede diseñarse para estimar con exactitud la valoración que daría un usuario a un elemento, mientras otro puede tener como principal objetivo el no proporcionar recomendaciones erróneas. Es decir puede haber múltiples tipos de medidas: que las recomendaciones cubran todo el espectro de elementos del conjunto (cobertura), que no se repitan, que sean explicables. \\
Sin embargo el principal objetivo de un sistema de recomendación no es directamente cuantificable: \textbf{la satisfacción del usuario.} En muchos casos conseguir un error cuadrático menor al elegir un algoritmo u otro no es apreciado por el usuario.
\paragraph{}
Sin embargo hay muchos otros parámetros que pueden influir en esa satisfacción: la sensación de credibilidad que ofrezca el sistema, la interfaz de usuario, la mejora del perfil al incluir nuevos votos.  En cualquier caso las medidas de precisión pueden dar una primera idea de qué tan bueno es el algoritmo principal del sistema de recomendación. Existen dos tipos de métodos de evaluación \citep{GalanNieto1994}: \textbf{métodos estadísticos} y \textbf{métricas de decisión.}

\subsection{Métodos estadísticos}
El parámetro de evaluación más utilizado es el \textbf{Error Medio Absoluto} (MAE, por sus siglas en inglés), que mide la desviación de las recomendaciones predichas y los valores reales. A menor MAE mejor predice el sistema las evaluaciones de los usuarios. \\
El MAE sin embargo puede dar una idea distorsionada del algoritmo para el caso de sistemas que tienen como objetivo encontrar una lista de buenos elementos recomendables. El usuario tan solo está interesado en los \textbf{N} primeros elementos de la lista. El error que se cometa al estimar el resto es indiferente. 
\paragraph*{}
Tampoco es recomendable en sistemas en los que la salida deba de ser una decisión binaria de si/no. Por ejemplo, con una escala de 1 a 10 si el umbral esta situado en 5, utilizando MAE se obtendrá un mayor error al errar de 9 a 5 que al errar de 5 a 4, lo cual no es cierto a la hora de medir el error de salida. Sin embargo es un tipo de error estadísticamente muy sencillo de comprender. 
\paragraph{}
MAE Posee muchas variaciones, como el \textbf{Error Cuadrático Medio} que persigue penalizar los mayores errores o el \textbf{Error Absoluto Normalizado} que facilita la tarea de establecer comparaciones entre pruebas con diferentes conjuntos de datos.

\subsection{Métricas de decisión}

Evalúan la efectividad en las predicciones que realiza un sistema ayudando al usuario a seleccionar los elementos de mayor calidad, es decir, con qué frecuencia el sistema realiza recomendaciones correctas. Para ello asumen que el proceso de predicción es binario: o el elemento recomendado agrada al usuario o no lo agrada. Sin embargo en la práctica se plantea el problema de evaluar esto. 
\paragraph{}
Una posible solución es la de dividir el conjunto de datos en dos conjuntos, entrenamiento y test. Se trabaja con el conjunto de entrenamiento y posteriormente se evalúa el resultado comparando las recomendaciones proporcionadas con las del conjunto de test. Aun siendo a veces útil esta técnica, hay que tener en cuenta que los resultados dependen fuertemente del porcentaje de elementos relevantes que el usuario haya votado. 

\subsubsection{Precisión y Recuperación}

Es la más conocida de las métricas decisión, se utiliza en muchos tipos de sistemas de recuperación de información \citep{Herlocker2004, Sarwar2001, Basu1998, Billsus1998}. \textbf{Precisión} es la probabilidad de que un elemento seleccionado sea relevante y \textbf{Recuperación} es la probabilidad de que sea seleccionado un elemento relevante, aunque en los sistemas de recomendación la \textbf{relevancia} es algo totalmente subjetivo. Esta métrica es mas intuitiva, puesto que establecer que un sistema tiene una precisión del 90 por ciento significa que de cada 10 elementos recomendados 9 serán buenas recomendaciones, algo que no queda claro proporcionando valores de error cuadrático medio.

\subsubsection{Característica de Funcionamiento del Receptor (ROC por sus siglas en inglés)}

 Es otra medida muy utilizada que proporciona una idea de la potencia de diagnóstico de un sistema de filtrado \citep{Hanley1982, Herlocker2004}. Las curvas ROC dibujan la  \textbf{especifidad} (Probabilidad de que un elemento malo del conjunto sea rechazado por el filtro) y la  \textbf{sensitividad} (probabilidad de que un elemento bueno al azar sea aceptado). Si un elemento es bueno o malo viene dado por las valoraciones de los usuarios. Las curvas se dibujan variando el umbral de predicción a partir del cual se acepta un elemento. El área bajo la curva se va incrementando cuando el filtro es capaz de retener más elementos buenos y menos malos.

\section{Evaluación del sistema}
Para evaluar el sistema de recomendación de restaurantes se calculó el error en las recomendaciones con el método de \textbf{Raíz del Error Cuadrático Medio} (RMSE por sus siglas en inglés).  En el sistema se definió una matriz de valoraciones reales (ratings), \textbf{RMSE} permite calcular el error entre un vector de predicciones \textit{$\displaystyle v_{1}= [x_{1,1},x_{1,2},...x_{1,n}]$} y un vector de valoraciones reales \textit{$\displaystyle v_{2}= [x_{2,1},x_{2,2},...x_{2,m}]$} tomando estos vectores se calculó el error:

\begin{equation}
\displaystyle RMSE(v_{1},v_{2}) = \sqrt{\sum_{i=1}^{n}(x_{1,i}-x_{2,i})^{2}\over n}
\end{equation}

\paragraph{}
Para el primer experimento se utilizó una base de datos de 50 usuarios y 12 restaurantes.  La matriz utilizada se muestra en el apéndice A.
\paragraph*{}
Se probó el algoritmo de filtrado colaborativo y generó recomendaciones con error bajo. Es importante mencionar que constantemente se estuvieron comparando las pruebas usando dos distancias diferentes, la \textbf{distancia euclidiana} y la \textbf{correlación de Pearson}, la diferencia en los resultados fue en la mayoría de los casos, de decimales.\\


En la tabla \ref{tab:TableRecom} se muestran los resultados de la recomendación utilizando ambas distancias. Cabe mencionar que las mejores predicciones fueron calculadas con la \textbf{correlación de Pearson} y esto se refleja en los datos de la tabla.
\paragraph*{}
\begin{table}[H]
	\caption{Recomendaciones con filtrado colaborativo.} \begin{center}
\linespread{0.9}\selectfont
\begin{tabular}{>{\footnotesize}p{2.0in}>{\footnotesize}p{2.0in}}
\hline
Correlación de Pearson &  \\
\hline                    
Predicción  & Restaurante    \\
\hline             
4.99  & La vuelta del rodeo  \\
3.73  & Yogurt Place         \\
3.56  & La Querencia         \\
2.65  & Akira Teriyaqui      \\
\hline
Distancia euclidiana       & \\
\hline                    
Predicción  & Restaurante    \\
\hline
5.0   & La vuelta del rodeo  \\
4.0   & Sótano Suizo         \\
3.56  & Yogurt Place         \\
2.66  & La Querencia         \\
2.59  & Akira Teriyaqui      \\
\hline
\end{tabular}
\label{tab:TableRecom}
\end{center}
\end{table}

En un segundo experimento, se utilizó \textbf{RMSE} para calcular el error, con los mismos datos utilizados para generar las recomendaciones de las tablas anteriores. La tabla \ref{tab:RmsdPearson} muestra los resultados utilizando la \textbf{correlación de Pearson} y la \ref{tab:RmsdEuclidiana} los resultados utilizando la \textbf{distancia euclidiana}.
\paragraph*{}
\begin{table}[H]
	\caption{Raíz del Error Cuadrático Medio con correlación de Pearson.} \begin{center}
\linespread{0.9}\selectfont
\begin{tabular}{>{\footnotesize}p{1.0in}>{\footnotesize}p{1.0in}>{\footnotesize}p{1.0in}>{\footnotesize}p{1.0in}>{\footnotesize}p{1.0in}}
\hline                    
Usuarios & Num. de votos & Calificación & Predicción & RMSE\\
\hline             
10 & 6.2 & 3.794871795 & 3.179487179 & 0.877058019\\
20 & 6.35 & 3.746987952 & 3.156626506 & 0.911770421\\
30 & 6.3 & 3.873015873 & 3.380952381 & 0.776643163\\
40 & 6.225 & 3.714285714 & 3.291666667 & 0.827359541\\
50 & 6.18 & 3.609756098 & 3.141463415 & 0.838116355\\
\hline
\end{tabular}
\label{tab:RmsdPearson}
\end{center}
\end{table}


\begin{table}[H]
	\caption{Raíz del Error Cuadrático Medio con distancia euclidiana.} \begin{center}
\linespread{0.9}\selectfont
\begin{tabular}{>{\footnotesize}p{1.0in}>{\footnotesize}p{1.0in}>{\footnotesize}p{1.03in}>{\footnotesize}p{1.0in}>{\footnotesize}p{1.0in}}
\hline                    
Usuarios & Num. de votos & Calificación & Predicción & RMSE\\
\hline             
10 & 6.2 & 3.794871795 & 3.179487179 & 0.877058019\\
20 & 6.35 & 3.746987952 & 3.168674699 & 0.87811408\\
30 & 6.3 & 3.873015873 & 3.404761905 & 0.77151675\\
40 & 6.225 & 3.714285714 & 3.30952381 & 0.83094897\\
50 & 6.18 & 3.609756098 & 3.2 & 0.832275752\\
\hline
\end{tabular}
\label{tab:RmsdEuclidiana}
\end{center}
\end{table}

Los resultados obtenidos comprueban que en este caso la diferencia entre una distancia y otra es pequeña.
El error más grande lo refleja la \textbf{correlación de Pearson} con 20 usuarios, pero si se analizan las predicciones se observa que  las predicciones de las distancias tienen una diferencia mínima de 0.012048193, entonces puede decirse que las predicciones son aceptables en ambos casos.\\
Esta observación también se refleja en la gráfica de representación de los datos mostrada en la figura \ref{fig:graficarmsd}:
\paragraph*{}
\begin{figure}[H]
\centering 
\framebox{
\includegraphics[width=0.8\textwidth]{./images/graficarmsd.pdf} 
} \caption{Gráfica de Raíz del Error Cuadrático Medio(RMSE).} 
\label{fig:graficarmsd} 
\end{figure}

La gráfica muestra que las variaciones entre una y otra no son significativas, sin embargo, la \textbf{correlación de Pearson} es más recomendada \citep{GalanNieto1994, Burke2002, Garciavaldez2009, Adomavicius2005}, es por esto que las recomendaciones están basadas en esta distancia. Partiendo del comportamiento de los datos, se puede decir que a partir de 30 usuarios el error en las predicciones empieza a disminuir. 
\paragraph{}
Posteriormente, se buscó obtener la \textbf{mejor} recomendación para el usuario utilizando los mismos datos que en los experimentos enteriores. Se hizo una prueba con los 30 usuarios, el promedio de votos por usuario fue 6. 
El objetivo fué comprobar si las recomendaciones obtenidas con dos operaciones de promedios resultaban diferentes, esto para identificar cuál podría mostrar un mejor resultado y optar por utilizar ese promedio en el sistema. 
\paragraph{}
Se experimentó con ambos promedios evaluando cada uno de los usuarios contenidos en la base de datos de pruebas y los resultados no variaron significativamente.
La prueba de evaluación nos dió los resultados mostrados en la tabla \ref{tab:promedios}, donde se observa una diferencia mínima y en algunos casos nula. Por lo tanto, se puede afirmar que ambos promedios son igualmente eficientes  para la recomendación en el sistema. 
\paragraph{}
Sin embargo, en la arquitectura del Sistema de Recomendación de Objetos de Aprendizaje \citep{Garciavaldez2009}, está definido el promedio ponderado, entonces el promedio que utilizó el sistema 
fue el mismo.
\paragraph*{}
Uno de los objetivos logrados en esta investigación fueron las recomendaciones con \textbf{mayor exactitud}, para este caso, basándonos en métodos estadísticos cuantificables. \\
Ciertamente, la precisión de una buena recomendación no da a los usuarios una experiencia \textbf{efectiva} y \textbf{satisfactoria}. Los sistemas de recomendación deben proporcionar no sólo precisión, sino también utilidad. Un sistema puede recomendar elementos muy populares  a usuarios tomando como métrica la popularidad pero esto no siempre resulta útil.
\paragraph*{}
Se podría en un futuro evaluar aspectos cualitativos del sistema, mediante métricas de decisión para obtener recomendaciones aceptables no sólo basadas en exactitud, sino también en aspectos cualitativos como la \textbf{satisfacción del usuario} que no puede ser minorizada si se considera que los sistemas de recomendación buscan sugerir al usuario como un \textit{amigo} lo haría.

\begin{table}[H]
	\caption{Comparación de las recomendaciones.} \begin{center}
\linespread{0.9}\selectfont
\begin{tabular}{>{\footnotesize}p{0.7in}>{\footnotesize}p{0.8in}>{\footnotesize}p{0.8in}>{\footnotesize}p{0.7in}>{\footnotesize}p{0.8in}>{\footnotesize}p{0.8in}}
\hline                    
Usuario & P.Ponderado & Promedio & Usuario & P.Ponderado & Promedio \\
\hline             
1    &    4.92    &    4.86  &  16  &    4.82    &    4.71 \\
2    &    4.89    &    4.86  & 17  &    4.66    &    4.58 \\
3    &    4.51    &    4.50  & 18  &    4.88    &    4.85 \\
4    &    4.90    &    4.86 &  19  &    4.86    &    4.85 \\
5    &    4.92    &    4.86  &  20  &    4.87    &    4.85 \\
6    &    4.89    &    4.84  &  21  &    4.89    &    4.85  \\
7    &    4.87    &    4.87  &  22  &    4.50    &    4.38 \\
8    &    4.91    &    4.86 &  23  &    4.86    &    4.86 \\
9    &    4.84    &    4.80 &  24  &    4.85    &    4.85 \\
10  &    4.87    &    4.84  &  25  &    4.54    &    4.61 \\
11  &    4.88    &    4.86  &  26  &    4.77    &    4.61 \\
12  &    4.57    &    4.35 &   27  &    4.83    &    4.71 \\
13  &    4.85    &    4.85 &   28  &    4.89    &    4.85 \\
14  &    4.86    &    4.86 &   29  &    4.85    &    4.85 \\
15  &    4.87    &    4.80 &   30  &    4.85    &    4.85 \\
\hline
\end{tabular}
\label{tab:promedios}
\end{center}
\end{table}




\chapter{Conclusiones y Trabajo Futuro}

\section{Conclusiones}

Las principales características del prototipo presentado son:
\begin{itemize}
\item Ofrece una mayor flexibilidad que los sistemas clásicos, ya que, permite a los usuarios que expresan sus preferencias mediante valoraciones lingüísticas.
\item Es capaz de realizar recomendaciones sólo si existe la información histórica o esta no se tiene que utilizar como es el caso del módulo basado en contenido. El uso de los algoritmos es necesario para recibir recomendaciones.
\item En el sistema se ha incluído un módulo basado en contenido para generar recomendaciones cuando no se requiera utilizar esta información histórica.
\item Es un prototipo que genera recomendaciones y su uso puede aprovecharse en  el sector turístico para facilitar la propagación de este tipo de herramientas en el mismo. 
\end{itemize}

\paragraph{}
Los sistemas de recomendación ofrecen recomendaciones personalizadas a sus usuarios, haciendo que el proceso de compra en tiendas virtuales  sea más rápido y personalizado. El objetivo es ofrecer a usuarios los productos que más les interesen o sean de su  \textbf{preferencia}. Para aprender estas preferencias los sistemas utilizan la información histórica del usuario, es decir, qué productos ha comprado o  evaluado. 
\paragraph{}
En este trabajo de investigación se explicó la arquitectura e implementación de un \textbf{sistema de recomendación de restaurantes híbrido.} Se utilizan algoritmos de filtrado colaborativo \citep{Segaran2007} y  basado en contenido para generar dos recomendaciones \citep{Burke2002}, una tercera recomendación es generada por un sistema difuso experto que toma parámetros identificados en cada restaurante para generar recomendaciones al usuario actual.
\paragraph{}
Para determinar las recomendaciones se involucran parámetros importantes a considerar para lograr una recomendación eficiente. Por un lado , el \textbf{ algoritmo de filtrado colaborativo} utiliza el mismo procedimiento que en otras investigaciones, aquí la diferencia la marca el conjunto de datos que se utilizan para obtener resultados. En este sistema se utilizó \textbf{información lingüística}, datos que de alguna manera son subjetivos y el sistema no está totalmente preparado para recibir este tipo de información, es posible que se necesite mayor granularidad en las funciones de membresía y agregar otros parámetros que también pueden ser importantes para un usuario al generar una recomendación.  
\paragraph{}
Por otro lado, las recomendaciones del \textbf{algoritmo basado en contenido} solamente estan basadas en la similaridad entre los restaurantes de base de datos,  aquí el algoritmo solo se basa en esta similaridad y no considera  la popularidad de cada restaurante entre los usuarios. Resultaría mucho mejor agregar este factor para mejorar el algoritmo, sin embargo, se obtienen recomendaciones aceptables.\\
\paragraph{}
Las técnicas de \textbf{lógica difusa} permiten implementar información imprecisa que maneja en la base de datos, de igual manera, esta información es recuperada por un sistema de inferencia que utiliza reglas difusas para asignar pesos a los algoritmos y generar una recomendación final mediante el promedio de pesos.
\paragraph*{}
Estos algoritmos están implementados en un prototipo que será utilizado para pruebas. Cada uno está generando recomendaciones de manera independiente, es decir, no influyen las recomendaciones de uno sobre el otro,  a su vez,  estas recomendaciones son mostradas al usuario a través de la interfaz diseñada en el prototipo.
\paragraph*{}
El método para evaluar el sistema fué la \textbf{Raíz del Error Cuadrático Medio (RMSE)} el cual permitió obtener las diferencias entre las predicciones de los algoritmos y las reales. Esto dió como resultado un error bajo considerando que el rango de valores es de 0 a 5 (ver tablas \ref{tab:RmsdPearson} y \ref{tab:RmsdEuclidiana}).
\paragraph{}
Los sistemas de recomendación han hecho progresos significativos en la década pasada, cuando numerosos sistemas  basados en el contenido, colaborativos y  métodos híbridos fueron propuestos y  han sido desarrollados. Sin embargo, a pesar de todos estos avances, la generación actual de sistemas de recomendación aún requiere nuevas mejoras para los métodos de recomendación más eficaz en una amplia gama de aplicaciones. 

\section{Trabajo Futuro}

Concluído el trabajo de investigación, se plantea en un futuro mejorarlo en los siguientes aspectos:
\begin{itemize} 
\item En la arquitectura considera tres módulos para generar recomendaciones al usuario activo. El \textbf{filtrado colaborativo} fue un buen algoritmo de recomendación, hizo predicciones con error bajo utilizando la \textbf{distancia euclidiana} y \textbf{correlación de Pearson}, estas distancias han sido muy utilizadas por otros investigadores por ser las que mejores resultados obtienen, sería una buena opción explorar otras y hacer más experimentos utilizando otras distancias.
\item En el caso del \textbf{algoritmo basado en contenido} se utilizó la \textbf{similaridad de cosenos}, fué sugerida como más efectiva para este tipo de algoritmos pero experimentar con otra opción para obtener una similaridad podría mostrar resultados inesperados.
\item El módulo de recomendaciones mediante el perfil de usuario podría agregarse en la arquitectura para generar recomendaciones basadas en el contenido del perfil de usuario, este caso podría utilizar \textbf{TF/IDF} (Frecuencia de Términos/Frecuencia del Documento Inversa) para analizar detalladamente el contenido de reseñas que el usuario hace sobre cada  restaurante que ha visitado.
\item Es conveniente agregar dentro del prototipo un \textbf{Sistema Geográfico Referenciado} como \citep{Espinilla2009} que permita la localización de cada restaurante de la ciudad y hacer recomendaciones basadas en la ubicación física del usuario activo, esto para tener más información sobre el usuario para generar recomendaciones que satisfagan sus expectativas.
\item La implementación de \textbf{RECOMET}, para obtener a partir del prototipo actual, un sistema de recomendación completamente funcional y de uso comercial en la ciudad.
\item Aplicar otras formas de obtención de información basándo las recomendaciones en una mayor cantidad de información de cada usuario.
\item Estudiar los \textbf{modelos de hibridación} más utilizados en otros sistemas con características similares a \textbf{RECOMET} con el objetivo de facilitar las recomendaciones y comparar con las obtenidas en el modelo actual del sistema.
\item Mayor flexibilidad en la definición de los \textbf{gustos} y \textbf{preferencias} del usuario, refinando el sistema para trabajar con mas datos lingüísticos.
\item Optimización de los sistemas difusos implementados para generar recomendaciones.
\item Evaluar el sistema utilizando \textbf{métricas de decisión} (Precisión y Recuperación, Idoneidad, ROC, etcétera) con el objetivo de mejorar la satisfacción del usuario en el sistema.
\item Crear una base de datos de restaurantes con información real ya que sólo se obtuvo información basada en Internet pero no fué comprobada su autenticidad.
\end{itemize}
Son muchas las mejoras que se pueden hacer al sistema, sin embargo, habría que analizar cuáles son las más convenientes y viables para su implementación, podría darse el caso que alguna empeore los resultados en las recomendaciones como el uso de otras distancias para obtener similaridad entre usuarios y restaurantes.

%\bibliographystyle{plain}
%\bibliography{biblio}
\begin{thebibliography}{99}

\bibitem{Acilar2009}  Acilar, A., Merve A.; Ahmed (2009). \textit{A collaborative filtering method based on artificial immune networks. Expert Sistems with applications,} 161.
\bibitem{Willey 2000}  Wiley, D. A. (2000). \textit{Connecting learning objects to instructional design theory: A definition, a metaphor, and a taxonomy. The Instructional Use of Learning Objects}, Bloomington: Association for Educational Communications and Technology, (pp. 3(23).



\end{thebibliography}

\appendix

\chapter{Programación del sistema}

\lstset{language=python}
\begin{lstlisting} [basicstyle=\scriptsize]	

#MODELO DE BASE DE DATOS EN DJANGO V1.2

from django.db import models
from models import *
import datetime
from django.utils.translation import get_date_formats
from django.utils import formats
from django.forms import ModelForm, TextInput, widgets
from django import forms
from django.contrib.auth.models import User

PRICE=(
    (1, 'very cheap'),
    (2, 'cheap'),
    (3, 'more and less'),
    (4, 'expensive'),
    (5, 'very expensive')
    )

LANG = (
    ('es', 'Espanol'),
    ('en', 'English'),
    ('fr', 'French'),
    ('jp', 'Japanese'),
   )
   
RATING=(
    (1, 'Awful'),
    (2, 'Bad'),
    (3, 'So so'),
    (4, 'Good'),
    (5, 'Excelent')
    )

class Cuisine(models.Model):
    cuisine = models.CharField(max_length=28)
    def __unicode__(self):
        return self.cuisine

class Atribute_Group(models.Model):
    group = models.CharField(max_length=32, null=True)
    def __unicode__(self):
        return self.group

class Atribute(models.Model):
    group = models.ForeignKey(Atribute_Group) 
    value = models.CharField(max_length=32)
    def __unicode__(self):
        return self.group.group
    
class RestaurantChain(models.Model):
    is_international = models.BooleanField()
    name = models.CharField(max_length=128)
    speciality = models.CharField(max_length=128)
    year_founded = models.PositiveSmallIntegerField(null=True)
    locations_worldwide = models.PositiveSmallIntegerField(null=True)
    
    def __unicode__(self):
        return self.name
    
class Recommender_rule(models.Model):
    rule = models.TextField()
    def __unicode__(self):
        return self.rule
    
class Item(models.Model): 
    rule = models.ForeignKey(Recommender_rule, null=True)
    def __unicode__(self):
        return self.rule.rule

class Restaurant(models.Model):
    item = models.OneToOneField(Item, unique=True) #Llave primaria
    name = models.CharField(max_length=255)
    address = models.TextField(max_length=512, null=True)
    description = models.TextField()
    needs_reservation = models.BooleanField()
    price_range = models.IntegerField(max_length=2,choices=PRICE, null=True)
    url = models.CharField(max_length=512, null=True)
    phone = models.CharField(max_length=18, null=True)
    hours = models.CharField(max_length=50, null=True)
    slug_name = models.SlugField(max_length=18, null=True)
    chain = models.ForeignKey(RestaurantChain, null=True)
    atribute = models.ManyToManyField(Atribute)
    cuisine =  models.ManyToManyField(Cuisine)
    lat = models.DecimalField(max_digits=8, decimal_places=5, null=True) 
    alt = models.DecimalField(max_digits=8, decimal_places=5, null=True)
    pub_date =  models.DateTimeField()    
    def __unicode__(self):
        return self.name

class Container(models.Model):
    container_name = models.CharField(max_length=64) 
    is_public = models.BooleanField()
    def __unicode__(self):
        return self.container_name

class Rating_Dimension(models.Model):
    dimension_name = models.CharField(max_length=64)
    priority = models.IntegerField(null=True)
    
    def __unicode__(self):
        return self.dimension_name

class UserProfile(models.Model):
    user = models.OneToOneField(User, unique=True)

    adress = models.CharField(max_length=512) 
    price = models.IntegerField(max_length=1,choices=PRICE)
    atribute = models.ManyToManyField(Atribute)
    cuisine = models.ManyToManyField(Cuisine)

    container = models.ManyToManyField(Container, through='Container_User')
    ratings = models.ManyToManyField(Item, through='Rating', related_name='ratings')
    reviews = models.ManyToManyField(Item, through='Review', related_name='reviews')
  
    def __unicode__(self):
        return self.user.first_name

class Rating(models.Model):
    id_item =  models.ForeignKey(Item) 
    user = models.ForeignKey(UserProfile) 
    id_dimension = models.ForeignKey(Rating_Dimension, null=True)
    date = models.DateTimeField()
    rating = models.PositiveSmallIntegerField(choices=RATING)
    interested = models.BooleanField()
    def __unicode__(self):
        return self.user.user.first_name
    
class Review(models.Model):
    item_reviewed = models.ForeignKey(Item)
    user = models.ForeignKey(UserProfile) #id_user
    title = models.CharField(max_length=128)
    review = models.TextField()
    status = models.PositiveSmallIntegerField(null=True)
    review_time = models.DateTimeField()
    #review_time = datetime.datetime.now()
    helpful_yes = models.IntegerField(max_length=128, null=True)
    helpful_no = models.IntegerField(max_length=128, null=True)
    language = models.CharField(max_length=2, choices=LANG)
    
    def __unicode__(self):
        return self.title
    
class Friends(models.Model):
    user = models.ForeignKey(UserProfile, related_name='id_user')
    id_friend = models.ForeignKey(UserProfile, related_name='friends') 
    time_added = models.DateTimeField(null=True)
    
    def __unicode__(self):
        return self.user.user.first_name

class Container_User(models.Model):
    id_container =  models.ForeignKey(Container)  
    user = models.ForeignKey(UserProfile)
    id_itemid_item = models.ForeignKey(Item)
    
    def __unicode__(self):
       return self.user.user.first_name

class Tag(models.Model):
    tag = models.CharField(max_length=18)
    id_item = models.ForeignKey(Item) 
    
    def __unicode__(self):
        return self.tag


#ALGORITMO DE FILTRADO COLABORATIVO.

def matuser(id):  
    r = Rating.objects.values_list('user','id_item','rating').filter(interested=0)
    list_t=[]
    if(r):
        for i in r:
            list_t.append(i)
    
    us_it = dict([(t[0],{}) for t in list_t])
    for t in list_t:
        us_it[t[0]][t[1]]=float(t[2])
    
    pearsonCorrelation = topMatches(us_it,id) 
    pr = getRecommendations(us_it,id,similarity=sim_pearson)
    der = getRecommendations(us_it,id,similarity=sim_distance)
    
    l_item=[]
    if len(pr)>3:
     for i in range(3):
        item = Restaurant.objects.get(item=pr[i][1])
        l_item.append(item)
    else:
     for i in range(len(pr)):
        item = Restaurant.objects.get(item=pr[i][1])
        l_item.append(item)
    return pr 


#ALGORITMO BASADO EN CONTENIDO.

def vecprofiles(id):
    rat2 = Rating.objects.all().filter(user=id,rating=5,interested=0)
    
    val2=[]
    if(rat2):       
        [val2.append(i.id_item.id) for i in rat2]
    
    profile = []
    for i in val2:
        ida2 = []
        idc2 = []
        res2 = []
        
        r = Restaurant.objects.get(item=i)    
        c1 = r.item.id
        
        binp2 = [0,0,0,0,0]
        for i in range(5):
            if(i==r.price_range):
                binp2[i]=1
            elif(r.price_range==5):
                binp2[4]=1
                
        c6 = binp2
        for a in r.atribute.all():
            id=a.id
            ida2.append(id)
           
        for c in r.cuisine.all():
            id=c.id
            idc2.append(id)

        ida2.sort()
        idc2.sort()
        
        binc2 = [0,0,0,0,0,0,0,0,0,0,0,0,0,0,0,0,0,0,0,0,
                 0,0,0,0,0,0,0,0,0,0,0,0,0,0,0,0,0] 

        bina2 = [0,0,0,0,0,0,0,0,0,0,0,0,0,0,0,0,0,0,0,0,
                 0,0,0,0,0,0,0,0,0,0,0,0,0,0,0,0,0,0,0,0,
                 0,0,0,0,0,0,0,0,0,0,0,0,0,0,0]
        
        for i in idc2:
            binc2[i]=1
        for i in ida2:
            bina2[i]=1
          
        c12 = bina2
        c13 = binc2 
        v2=binp2+bina2+binc2
        
        res2.append(c1)
        res2.append(v2)   
        profile.append(res2)
        
    p_it = dict([(t[0],{}) for t in profile])
    for t in profile:
        p_it[t[0]] = (t[1])
        
    rat = Rating.objects.all().filter(interested=0)
    
    items = []
    [items.append(item) for item in rat if item not in rat2 if item.rating==5]
    
    val=[]
    if(rat):       
        for i in items:
            if i.id_item.id not in val2:
                val.append(i.id_item.id)

    rat3 = Rating.objects.all().filter(user=id,interested=0)
    l=[]
    [(l.append(i.id_item.id))for i in rat3]
    for item in l:
        if item not in val:continue
        val.remove(item)
    
    for i in range(len(val)):
        if(i<len(val)):
            for j in range(len(val)):
                if(j<len(val)):
                    if(i!=j):
                        if(val[i]==val[j]):
                            val.remove(val[j])
  
    profilei = []
    for i in val:
        ida = []
        idc = []
        res = []
        
        r = Restaurant.objects.get(item=i)
        c1 = r.item.id
        
        binp = [0,0,0,0,0]
        for i in range(5):
            if(i==r.price_range):
                binp[i]=1
            elif(r.price_range==5):
                binp[4]=1         
        c6 = binp
        
        for a in r.atribute.all():
            id=a.id
            ida.append(id)
           
        for c in r.cuisine.all():
            id=c.id
            idc.append(id)
        
        ida.sort()
        idc.sort()
        
        binc = [0,0,0,0,0,0,0,0,0,0,0,0,0,0,0,0,0,0,0,
                0,0,0,0,0,0,0,0,0,0,0,0,0,0,0,0,0,0] 

        bina = [0,0,0,0,0,0,0,0,0,0,0,0,0,0,0,0,0,0,0,
                0,0,0,0,0,0,0,0,0,0,0,0,0,0,0,0,0,0,0,
                0,0,0,0,0,0,0,0,0,0,0,0,0,0,0,0,0]
        
        for i in idc:
            binc[i]=1
            
        for i in ida:
            bina[i]=1
         
        c12 = bina 
        c13 = binc 
        v=binp+bina+binc
        
        res.append(c1)
        res.append(v)
        profilei.append(res)
        
    tp_it = dict([(t[0],{}) for t in profilei])
    for t in profilei:
       tp_it[t[0]]=(t[1])
       
    sim_cos = simCosenoItems(p_it, tp_it)
    
    t=[]
    for i in range(len(sim_cos)):
        tot=len(sim_cos[i])-2
        for j in range(tot):
            if(j<tot):
                if(j%3==0):
                    l_dis=[]
                    l_dis.append(sim_cos[i][j])
                    l_dis.append(sim_cos[i][j+1])
                    l_dis.append(sim_cos[i][j+2])
                    t.append(l_dis)
    
    hd=[]
    for i in t:
        if i[2]>0.7:
            hd.append(i)

    lis = dict([(l[0],{}) for l in hd])
    for l in hd:
        lis[l[0]][l[1]] = (l[2])

    bc_item=[]
    if len(hd)>3:
        for i in range(3):
            item = Restaurant.objects.get(item=hd[i][1])
            bc_item.append(item)
    else:
        for i in range(len(hd)):
            item = Restaurant.objects.get(item=hd[i][1])
            bc_item.append(item)
    
    return profile, val2, hd


#SISTEMA DIFUSO EXPERTO.

def fisExpert(id):   
    profile, val2, hd = vec_profiles(id) 
    prices=[] 
    it_prices=[] 
    
    for i in range(len(profile)):
        for j in range(4):
            if profile[i][1][j]==1:
                lp =[]
                lp.append(profile[i][0])
                lp.append(j+1)
                prices.append(lp)
                it_prices.append(profile[i][0])
                
    rtg = Rating.objects.values_list('user','id_item','rating').filter(interested=0)
    
    average=[]
    for i in val2:
        pin=[]
        c=0
        sum=0
        av=0
        for r in rtg: 
            if r[1]==i:
                c = c + 1
                sum = sum + float(r[2])
        if (c<1):continue 
        av=sum/c
        pin.append(i)
        pin.append(av)
        average.append(pin) 
      
    votes = []
    for i in val2:
        vo =[]
        c=0
        for r in rtg: 
            if r[1]==i:
                c=c+1
        vo.append(i)
        vo.append(c)
        votes.append(vo) 

    resProfile = []
    
    for i in range(len(val2)):
        a=0
        b=0
        c=0
        p=[]
        for j in range(len(average)):
            if(average[j][0]==val2[i]):
                p.append(val2[i])
                p.append(float(average[j][1]))
                a=1
        if(a==0):
            p.append(0.0)
            
        for k in range(len(prices)):
            if (prices[k][0]==val2[i]):
                p.append(prices[k][1])
                b=1
        if(b==0):
            p.append(0.0)
            
        for m in range(len(votes)):
            if(votes[m][0]==val2[i]):
                p.append(float(votes[m][1]))
                c=1
        if(c==0):
            p.append(0.0)
        
        resProfile.append(p)
    
    recomExpert=[]
    
    for p in resProfile:
        recom = expert.eval(p[1],p[2],p[3])
        if(recom>2.5):
            re=[]
            re.append(p[0])
            re.append(recom)
           
            recomExpert.append(re)
    
    return recomExpert
\end{lstlisting}

%\chapter{Matriz de valoraciones}
\subsubsection{Matriz de valoraciones}

\begin{table}[H]
	\caption{Matriz de valoraciones de usuarios.} \begin{center}
\linespread{0.9}\selectfont
\renewcommand {\arraystretch}{0.5}
\begin{tabular}{>{\footnotesize}p{0.4in}>{\footnotesize}p{0.2in}>{\footnotesize}p{0.2in}>{\footnotesize}p{0.2in}>{\footnotesize}p{0.2in}>{\footnotesize}p{0.2in}>{\footnotesize}p{0.2in}>{\footnotesize}p{0.2in}>{\footnotesize}p{0.2in}>{\footnotesize}p{0.2in}>{\footnotesize}p{0.2in}>{\footnotesize}p{0.2in}>{\footnotesize}p{0.2in}}
\hline   
 {\tiny Usuarios} &  {\tiny $I_1$} &  {\tiny $I_2$} &   {\tiny $I_3$} &   {\tiny $I_4$} &  {\tiny $I_5$} &  {\tiny $I_6$} &   {\tiny $I_7$} &  {\tiny  $I_8$ }&  {\tiny  $I_9$} & {\tiny  $I_{10}$} & {\tiny  $I_{11}$} &  {\tiny $I_{12}$}\\
\hline           
 {\tiny $U_1$} & {\tiny 5.0} & {\tiny 5.0}  & {\tiny 4.0}  & {\tiny 2.0}  & {\tiny 3.0}  & {\tiny 4.0}  &  &  &  &  &  & \\
 {\tiny $U_2$} &  {\tiny 4.0} &  {\tiny 5.0}  &  {\tiny 4.0}  &  &  {\tiny 3.0}  &  {\tiny 5.0}  &  {\tiny 1.0}  &  {\tiny 3.0}  &  &  &  & \\
 {\tiny $U_3$} &  {\tiny 3.0}  &  {\tiny 5.0}  & {\tiny 4.0}  &  {\tiny 3.0}  &  {\tiny 3.0}  &  &  {\tiny 2.0}  &  {\tiny 3.0}  &  &  &  & \\
 {\tiny $U_4$} &  &  &  &  {\tiny 3.0}  &  {\tiny 3.0}  &  &  {\tiny 2.0}  &  {\tiny 3.0}  &  &  &  & \\
 {\tiny $U_5$} &  {\tiny 4.0}  &  {\tiny 4.0}  &  {\tiny 4.0}  &  {\tiny 3.0}  &  &  {\tiny 5.0}  &  &  {\tiny 3.0}  &  &  &  & \\
 {\tiny $U_6$} &  {\tiny 3.0}  &  {\tiny 5.0}  &  {\tiny 4.0}  &  {\tiny 3.0}  &  {\tiny 3.0}  &  {\tiny 5.0}  &  {\tiny 3.0}  &  {\tiny 4.0}  &  &  &  & \\
 {\tiny $U_7$} &  {\tiny 3.0}  &  &  {\tiny 4.0}  &  {\tiny 3.0}  &  {\tiny 3.0}  &  {\tiny 5.0}  &  &  &  &  &  & \\
 {\tiny $U_8$} &  &  {\tiny 5.0}  &  {\tiny 5.0}  &  {\tiny 2.0}  &  {\tiny 4.0}  &  {\tiny 5.0}  &  {\tiny 2.0}  &  {\tiny 5.0}  &  &  &  & \\
 {\tiny $U_9$} &  {\tiny 3.0}  &  {\tiny 4.0}  &  {\tiny 4.0}  &  {\tiny 2.0}  &  {\tiny 4.0}  &  &  {\tiny 3.0}  &  {\tiny 5.0}  &  &  &  & \\
 {\tiny $U_{10}$} &  {\tiny 3.0}  &  {\tiny 5.0}  &  {\tiny 5.0}  &  &  {\tiny 4.0}  &  &  &  {\tiny 4.0}  &  &  &  & \\
 {\tiny $U_{11}$} &  {\tiny 3.0}  &  {\tiny 4.0}  &  {\tiny 4.0}  &  {\tiny 4.0}  &  {\tiny 4.0}  &  {\tiny 5.0}  &  {\tiny 3.0}  &  {\tiny 4.0}  &  &  &  & \\
 {\tiny $U_{12}$} &  &  & {\tiny 5.0}  &  {\tiny 4.0}  &  {\tiny 3.0 } &  {\tiny 5.0}  &  {\tiny 2.0}  &  {\tiny 4.0}  &  &  &  & \\
 {\tiny $U_{13}$} &  {\tiny 5.0}  &  {\tiny 5.0}  &  &  {\tiny 2.0}  &  {\tiny 5.0}  &  {\tiny 5.0}  &  {\tiny 2.0}  &  {\tiny 5.0}  &  &  &  & \\
 {\tiny $U_{14}$} &  &  & {\tiny 3.0}  &  {\tiny 3.0}  &  {\tiny 5.0}  &  {\tiny 4.0}  &  {\tiny 3.0}  &  &  &  &  & \\
 {\tiny $U_{15}$} &  {\tiny 4.0}  &  {\tiny 4.0}  &  {\tiny 3.0}  &  {\tiny 4.0}  &  {\tiny 3.0}  &  {\tiny 4.0}  &  &  {\tiny 5.0}  &  &  &  & \\
 {\tiny $U_{16}$} &  {\tiny 3.0}  &  {\tiny 4.0}  &  {\tiny 3.0}  &  {\tiny 2.0}  &  {\tiny 3.0}  &  {\tiny 4.0}  &  &  &  &  &  & \\
 {\tiny $U_{17}$} &  {\tiny 3.0}  & {\tiny 4.0}  &  {\tiny 3.0}  &  {\tiny 3.0}  &  {\tiny 3.0}  &  &  {\tiny 5.0}  &  {\tiny 3.0}  &  &  &  & \\
 {\tiny $U_{18}$} &  &  {\tiny 3.0}  &  {\tiny 3.0}  &  {\tiny 3.0}  &  {\tiny 3.0}  &  {\tiny 4.0}  &  &  {\tiny 3.0}  &  &  &  & \\
 {\tiny $U_{19}$} &  {\tiny 1.0}  &  {\tiny 5.0}  &  {\tiny 3.0}  &  {\tiny 3.0}  &  {\tiny 3.0}  &  {\tiny 4.0}  &  &  {\tiny 3.0}  &  &  &  & \\
 {\tiny $U_{20}$} &  &  &  {\tiny 3.0}  &  {\tiny 2.0}  &  {\tiny 3.0}  &  {\tiny 4.0}  &  {\tiny 3.0}  &  {\tiny 3.0}  &  &  &  & \\
 {\tiny $U_{21}$} &  &  {\tiny 3.0}  &  {\tiny 3.0}  &  {\tiny 4.0}  &  {\tiny 3.0}  &  {\tiny 5.0}  &  {\tiny 2.0}  &  &  &  &  & \\
 {\tiny $U_{22}$} &  {\tiny 3.0}  &  {\tiny 5.0}  &  {\tiny 3.0}  &  {\tiny 5.0}  &  {\tiny 3.0}  &  {\tiny 4.0}  &  &  &  &  &  & \\
 {\tiny $U_{23}$} &  {\tiny 4.0}  &  {\tiny 5.0}  &  &  {\tiny 4.0}  &  {\tiny 3.0}  &  {\tiny 5.0}  &  &  &  &  &  & \\
 {\tiny $U_{24}$} &  &  &  {\tiny 4.0}  &  {\tiny 4.0}  &  {\tiny 3.0}  &  {\tiny 4.0}  &  {\tiny 2.0}  &  {\tiny 3.0}  &  &  &  & \\
 {\tiny $U_{25}$} &  {\tiny 4.0}  &  {\tiny 5.0}  &  {\tiny 4.0}  & {\tiny 4.0}  &  {\tiny 4.0}  &  {\tiny 5.0}  &  {\tiny 1.0}  &  {\tiny 5.0}  &  &  &  & \\
 {\tiny $U_{26}$} &  &  {\tiny 5.0}  &  {\tiny 4.0}  &  {\tiny 4.0}  &  {\tiny 4.0}  &  {\tiny 4.0}  &  {\tiny 2.0}  &  {\tiny 5.0}  &  &  &  &  {\tiny 5.0} \\
 {\tiny $U_{27}$} &  & {\tiny 5.0}  &  {\tiny 4.0}  &  {\tiny 3.0}  &  {\tiny 4.0}  &  {\tiny 5.0}  &  {\tiny 3.0}  &  &  &  &  & \\
 {\tiny $U_{28}$} &  & {\tiny 5.0}  &  &  {\tiny 3.0}  &  {\tiny 4.0}  &  {\tiny 4.0}  &  &  {\tiny 5.0}  &  &  &  & \\
 {\tiny $U_{29}$} &  {\tiny 2.0}  &  &  {\tiny 5.0}  &  {\tiny 3.0}  &  {\tiny 4.0}  &  {\tiny 5.0}  &  {\tiny 1.0}  &  {\tiny 5.0}  &  &  &  & \\
 {\tiny $U_{30}$} &  & {\tiny 4.0}  &  {\tiny 4.0}  &  {\tiny 4.0}  &  &  {\tiny 4.0}  &  &  {\tiny 2.0}  &  &  &  & \\
 {\tiny $U_{31}$} &  {\tiny 3.0}  &  {\tiny 4.0}  &  {\tiny 4.0}  &  {\tiny 4.0}  &  {\tiny 2.0}  &  &  {\tiny 3.0}  &  {\tiny 3.0}  &  &  &  & \\
 {\tiny $U_{32}$} &  {\tiny 3.0}  &  {\tiny 3.0}  &  &  {\tiny 4.0}  &  {\tiny 2.0}  &  &  &  &  &  &  & \\
 {\tiny $U_{33}$} &  &  {\tiny 5.0}  &  {\tiny 4.0}  &  &  {\tiny 3.0}  &  {\tiny 5.0}  &  {\tiny 2.0 } &  {\tiny 3.0}  &  &  &  & \\
 {\tiny $U_{34}$} &  {\tiny 5.0}  &  {\tiny 2.0}  &  {\tiny 3.0}  &  {\tiny 4.0}  &  {\tiny 2.0}  &  {\tiny 5.0}  &  {\tiny 3.0}  &  &  &  {\tiny 3.0}  &  & \\
 {\tiny $U_{35}$} &  &  {\tiny 3.0}  &  &  {\tiny 5.0}  &  {\tiny 3.0}  &  {\tiny 5.0}  &  &  &  {\tiny 2.0}  &  &  & \\
 {\tiny $U_{36}$} &  {\tiny 2.0}  &  &  &  &  {\tiny 2.0}  &  {\tiny 3.0}  &  {\tiny 2.0}  &  &  &  &  &  {\tiny 5.0} \\
 {\tiny $U_{37}$} &  {\tiny 3.0}  &  {\tiny 3.0}  &  {\tiny 4.0}  &  {\tiny 3.0} &  {\tiny 3.0}  &  {\tiny 3.0}  &  &  {\tiny 3.0}  &  &  &  & \\
 {\tiny $U_{38}$} &  &  {\tiny 4.0}  &  {\tiny 3.0}  &  {\tiny 2.0}  &  &  {\tiny 3.0}  &  &  {\tiny 3.0}  &  &  &  &  {\tiny 5.0} \\
 {\tiny $U_{39}$} &   {\tiny 4.0} &  &  {\tiny 3.0}  &  {\tiny 4.0}  &  {\tiny 2.0}  &  {\tiny 5.0}  &  {\tiny 3.0}  &  {\tiny 3.0}  &  &  &  & \\
 {\tiny $U_{40}$} &  {\tiny 5.0}  &  &  {\tiny 5.0}  &  {\tiny 2.0}  &  {\tiny 2.0}  &  {\tiny 4.0}  &  &  &  &  &  & \\
 {\tiny $U_{41}$} &  &  &  &  {\tiny 3.0}  &  {\tiny 2.0}  &  {\tiny 5.0}  &  &  {\tiny 3.0}  &  {\tiny 4.0}  &  &  & \\
 {\tiny $U_{42}$} &  {\tiny 3.0}  &  {\tiny 2.0}  &  &  {\tiny 3.0}  &  &  {\tiny 3.0}  &  {\tiny 2.0}  & {\tiny 3.0}  &  &  {\tiny 2.0}  &  & \\
 {\tiny $U_{43}$} &  & {\tiny 2.0}  &  &  {\tiny 3.0}  & {\tiny 3.0}  &  {\tiny 5.0}  &  &  {\tiny 3.0}  &  &  &  & \\
 {\tiny $U_{44}$} &  &  &  {\tiny 3.0}  &  {\tiny 2.0}  &  {\tiny 3.0}  &  {\tiny 4.0}  &  {\tiny 5.0}  &  {\tiny 3.0}  &  &  &  & \\
 {\tiny $U_{45}$} &  {\tiny 5.0}  &  {\tiny 2.0}  &  {\tiny 3.0}  &  &  &  {\tiny 3.0}  &  {\tiny 4.0}  &  &  &  &  & \\
 {\tiny $U_{46}$} & {\tiny 4.0}  &  {\tiny 2.0}  &  {\tiny 3.0}  &  {\tiny 4.0}  &  {\tiny 3.0}  &  {\tiny 5.0}  &  &  {\tiny 3.0}  &  &  &  & \\
 {\tiny $U_{47}$} &  &  &  {\tiny 3.0}  &  {\tiny 2.0}  &  {\tiny 3.0}  &  &  {\tiny 2.0}  &  {\tiny 3.0}  &  &  &  & \\
 {\tiny $U_{48}$} &  &  {\tiny 2.0}  &  {\tiny 3.0}  &  &  {\tiny 3.0}  &  &  {\tiny 2.0}  &  {\tiny 4.0}  &  &  {\tiny 4.0}  &  & \\
 {\tiny $U_{49}$} &  {\tiny 5.0}  &  &  {\tiny 3.0}  &  {\tiny 4.0}  &  {\tiny 3.0}  &  {\tiny 4.0}  &  {\tiny 3.0}  &  &  &  &  & \\
 {\tiny $U_{50}$} &  &  &  &  &  &  {\tiny 5.0}  &  &  {\tiny 4.0}  &  {\tiny 2.0}  &  {\tiny 2.0}  &  {\tiny 4.0}  &  {\tiny 5.0} \\
\hline
\end{tabular}
\label{tab:matriz}
\end{center}
\end{table}

\end{document}


